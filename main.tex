\documentclass[11pt, a4paper]{article}

\usepackage[utf8]{inputenc}
\usepackage[T1]{fontenc}
\usepackage{helvet}
\usepackage{multirow, hyphenat, blindtext, hyperref, framed, textcomp, xcolor, fancyhdr, titling}
\usepackage[left=3cm, right=3cm]{geometry}
\usepackage[stable]{footmisc}
\usepackage[parfill]{parskip} % empty line instead of indent
\usepackage[section]{minted}

\renewcommand{\familydefault}{\sfdefault}

\pagestyle{empty}

\fancypagestyle{firstpage}{
  \fancyfoot{}
  \renewcommand{\headrulewidth}{0pt}
}

\fancypagestyle{content} {
  \lhead{Michael König}
  \rhead{\thepage}
  \fancyfoot{}
}

\hypersetup{
    colorlinks=true, 
    linkcolor=,
    urlcolor=blue,
}

\newmintedfile[mypy]{python}{
fontfamily=tt,
linenos=true,
numberblanklines=true,
numbersep=5pt,
gobble=0,
frame=single,
framerule=0.4pt,
framesep=2mm,
funcnamehighlighting=true,
tabsize=2,
obeytabs=false,
mathescape=false
showspaces=false,
showtabs =false,
texcl=false,
}

\newmintedfile[mytext]{text}{
fontsize=\small
}

\renewcommand\maketitlehooka{\null\mbox{}\vfill}
\renewcommand\maketitlehookd{\vfill\null}
\renewcommand{\abstractname}{Introduction}

\title{Introduction to Python}
\author{Michael König}
\date{\today}

\begin{document}
  \begin{titlingpage}
    \maketitle
    \thispagestyle{firstpage}
    
    \begin{abstract}
      \noindent
      Python became one of the most popular programming languages in recent years and is still growing. It makes programming easier and allows the programmer to achieve much in short syntax.\\
      This introduction assumes knowledge in programming and is not an introduction to programming general.
    \end{abstract}
  \end{titlingpage}

  \tableofcontents
  \newpage
  \pagestyle{content}
  \setcounter{page}{1}

  \section{Common Terms}
    Before getting into programming and the syntax itself I'd like to explain some common terms you'll hear all over the web and maybe work with:
    \begin{enumerate}
        \item \textbf{CPython:}\\
            CPython is a Python interpreter, so it's an implementation of Python.
            It's the most common interpreter available.
        \item \textbf{PyPy\footnote{PyPy is a Python interpreter written in Python}:}\\
            Offering better performance, PyPy is an interpreter as well and often preferred over CPython when performance is important.
        \item \textbf{IPython:}\\
            IPython is an interactive Python-Shell.
        \item \textbf{Jupyter Notebook:}\\
            You can think of Jupyter Notebooks as a document storing both text cells and code cells - enabling you to describe your code and plot graphs. They're often used for Data Science.
            You can either install Jupyter on your local machine or use cloud-hosted solutions like \href{https://colab.research.google.com/}{Google Colab}.
        \item \textbf{PEP \& PEP8:}
            The Python Enhancement Proposals - short PEP - are documents with different contents, mainly describing language specifications and features Python offers.\\
            One of the most well-known is PEP8 - the style guide for Python.\\
            Depending on the programming language you're coming from you might have seen different people using different code formatting.
            Of course, it's your choice how you format your code - but PEP8 is basically used everywhere and I highly encourage you to use it, too.
        \end{enumerate}
        
    \subsection{Common Modules From The Standard Library}
        The standard library consists of many different modules and the following is only a brief overview of the most used ones.\\
        If you're just getting into Python you shouldn't worry about them for you - just remember they exist so you can look them up once you need them.
        
        \begin{enumerate}
            \item \textbf{os:} interaction with the operating system like accessing the file system
            \item \textbf{sys:} information about the file system
            \item \textbf{re:} working with regular expressions
            \item \textbf{tkinter:} creation of graphical user interfaces
            \item \textbf{itertools:} provides efficient iterators
            \item \textbf{functools:} provides higher-order functions
            \item \textbf{pickle:} for serialization
            \item \textbf{argparse:} makes it easy to parse and specify command line arguments
        \end{enumerate}
    
    \subsection{Common Third-Party Modules}
        There are different ways to install third-party Python modules, the most common way is by using pip, Python's package manager which is part of your Python distribution (if you've downloaded it from \href{https://python.org}{python.org})\\
        The following are some of the most widely used third-party modules and I recommend you to look into them:
        \begin{enumerate}
            \item \textbf{SciPy:} used for scientific and technical computing; contains different modules, one of the most popular ones is NumPy. 
            \item \textbf{NumPy:} provides fast (multi-dimensional) arrays with many handy features and functions. One NumPy array has exactly one data type.
            \item \textbf{pandas:} mainly used because of their dataframes; you think of dataframes as (e.g. Excel) spreadsheets, so two-dimensional arrays where every column can have a different data type.
            \item \textbf{matplotlib:} used for plotting graphs (e.g. line plots, scatter plots, histograms, heat maps)
        \end{enumerate}
    

\newpage
\section{Basic Syntax}

  \subsection{If/Else \& For In}

  \subsection{Inline If/Else}

  \subsection{Chaining Conditions}

  \subsection{Strings And Comments}

    \subsection{Modulo Operator}

    \subsubsection{Format Method}

    \subsubsection{F-Strings}
  
  \subsection{Exception Handling}
\newpage
\section{Imports}

  Once you've installed a module (or it's part of the standard library) you can
  easily import it using the import statement.

  Let's consider numpy as an example:\\
  \mintinline{python}{import numpy} enables you to access numpy's objects and modules, 
  for instance the \textit{ones} method (which gives you an array filled with zeros)
  by calling \mintinline{python}{numpy.ones((2,2))}.

  Most of the time, however, you'll see people use \textit{np} as an alias for numpy.
  Just type \mintinline{python}{import numpy as np} and you're able to call the
  \textit{ones} method with \mintinline{python}{np.ones((2,2))}

  If you just need some specific objects or methods, like the \textit{zeros} or \textit{ones} method,
  you can import them by using \mintinline{python}{from numpy import zeros, ones}.
  You can then access the methods by simply saying
  \mintinline{python}{ones((2,2))} and there's no need for the 'numpy' at front.

  Newcomers often tend to use \mintinline{python}{from numpy import *} which
  enables you to use any method or object from the module without prefixing
  it with the module name at front or importing it explicitly by using its name.\\
  However, you should \textbf{never use a \mintinline{python}{from x import *}}
  as it pollutes your namespace, meaning you've got a problem if
  two or more modules have members with the same name.

  Consider the following example:
  \mypy{content/partials/imports/import_star.py}
  \mytext{content/partials/imports/import_star.out}

  In both modules a variable \textit{some\_var} was declared and set to different values.
  As you can see in the output, some\_var from b overwrote some\_var from a.
  To avoid these kind of issues, just import modules under a short alias, like
  np for numpy, pd for pandas or tk for tkinter.

  All the methods introduced above also work for submodules, so to use the \textit{pyplot} submodule
  from \textit{matplotlib} as \textit{plt} you can import it using
  \mintinline{python}{import matplotlib.pylot as plt}.
\newpage
\section{Functions}

  I assume you already know about functions; this is what their basic syntax looks like in Python:

  \mypy{content/partials/functions/basic_syntax.py}
  \mytext{content/partials/functions/basic_syntax.out}

  You can also set default parameters by saying
  \mintinline{python}{def my_function(x=1, y=1):}
  so you can call the function with zero, one or two arguments.
  Parameters with a default value are \textbf{optional arguments}, those without are called
  \textbf{positional arguments}.

  Note that \textbf{optional parameters} must be the last arguments in the function definition.
  So \mintinline{python}{def my_function(x=1, y):} is not a valid definition.

  Furthermore, you can \textbf{specify the desired data types} of the function arguments with
  \mintinline{python}{def my_function(x: int, y: int) -> int:}.
  However, everyone's still able to pass variables of any kind to that function,
  meaning \textbf{the data type is not enforced} by any means but just
  for documentation and readability!

  \mypy{content/partials/functions/parameters.py}
  \mytext{content/partials/functions/parameters.out}

  If you wanted to return two items you'd usually return a tuple instead of a list like I did,
  so we'll revisit this function later and improve it when we talk about tuple unpacking.

  As you can see on line 9 you can \textbf{set a parameter by its variable name}
  instead of position.
  This is especially useful when working with functions which can have a huge number of optional
  arguments and you need to only set some specific ones.
  
  \subsection{Arbitrary Argument Lists}

    Sometimes you want to be able to call a function with any number of variables;
    in Python, this is done with *args and you might know this concept from other programming
    languages as 'varargs' (variables arguments).

    Let's take a function \mintinline{python}{my_sum()} as an example:

    \mypy{content/partials/functions/my_sum.py}
    \mytext{content/partials/functions/my_sum.out}

    The function \mintinline{python}{my_sum()} stores all the values it gets in the *args list.
    The * denotes that the variables called 'args' takes all the spare values; The * variables must
    be written after all the positional and optional arguments.
    it is later accessed by just using \mintinline{python}{args}, not \mintinline{python}{*args}!
  
  \subsection{Keyword Arguments}

  \subsection{Lambdas}
    Lambdas are a shorthand for one-line functions and created using the
    \mintinline{python}{lambda} keyword. They receive variables and their function body consists
    of only one term, which value also gets returned, therefore an explicit
    \mintinline{python}{return} is not needed.

    \mypy{content/partials/functions/lambda.py}
    \mytext{content/partials/functions/lambda.out}

  \subsection{Tuple Unpacking}
\newpage
\section{Lists, Sets \& Dictionaries}

  \subsection{List Slicing}

  \subsection{Useful Functions For Iterating}
\newpage
\section{Generators}
    Your memory probably can't handle a list of hundreds of thousands of elements.
    Therefore we use generators for tasks like this instead.
    Generators calculate and deliver a value when it's needed.
    
    Consider the following example:

    \mypy{content/partials/generators/gen.py}
    \mytext{content/partials/generators/gen.out}
    
    At first, we define a new generator.
    The only thing keeping it apart from a function is the difference that functions return a
    value whereas generators \textbf{yield} a value.
    
    Upon calling \mintinline{python}{gen()} on line 5 we get a generator object.
    
    When we call \mintinline{python}{next(g)} (with g being the generator object) the generator
    function executes till it reaches the next \mintinline{python}{yield} statement.\\
    Once a value is being \mintinline{python}{yielded}, the generator function stops and returns
    the value yielded.\\
    At the next \mintinline{python}{next(g)} the generator function gets resumed till it reaches
    its next \mintinline{python}{yield}.
    
    In this case, the generator would be able to provide \( 10^{10000} \) values.
    
    
\newpage
\section{Comprehensions}

    Often you want to create basic collections following a simple pattern, let's say a list
    containing all the squares of the numbers 1 to 10. You had to create an empty list and then
    fill it within a for loop. \\
    To do this with less code, Python provides different kinds of comprehensions to create
    collections in one line. \\
    The list comprehension will probably be your most used one:

    \subsection{List Comprehensions}

        Let's take the example I just talked about and code it with the knowledge we
        acquired so far:

        \mypy{content/partials/comprehensions/normal_list.py}
        \mytext{content/partials/comprehensions/normal_list.out}

        To achieve the same with a list comprehension:

        \mypy{content/partials/comprehensions/list_comp.py}
        \mytext{content/partials/comprehensions/list_comp.out}

        The \textbf{first expression} - in this case \mintinline{Python}{x**2} -
        is \textbf{added to the list} and this is repeated for every \mintinline{Python}{x} in
        \mintinline{Python}{range(1,11)}, so the numbers 1-10 (as the end of the range -
        the second parameter - is exclusive). \\
        You can also use nested for loops:
        \mintinline{Python}{l = [x for x in item for item in list]}

        \subsubsection{Filtering Elements}

            If you only want to add an element to the list if a certain criteria is met, you can add
            an \mintinline{Python}{if} statement \textbf{at the end of the comprehension}:

            \mypy{content/partials/comprehensions/list_comp_if.py}
            \mytext{content/partials/comprehensions/list_comp_if.out}

        \subsubsection{Ternary Operator}

            You can use the ternary operator in the first expression of the comprehension to decide
            which value to add to the list based on some condition.

            Note that the ternary operator - the \textbf{if at the start of the comprehension} -
            decides on which value to add to the list and the if at the end of the comprehension
            decides on whether the element should even be added to the list or not.

            \mypy{content/partials/comprehensions/list_comp_tern.py}
            \mytext{content/partials/comprehensions/list_comp_tern.out}

            As you can see, the comprehension got quite long and therefore I decided to put
            a line break in.

            \textbf{If your comprehension gets complex you should definitely consider using a normal loop
            instead}, as it's probably easier to read.

    \subsection{Set Comprehensions}

        Set comprehension work just like list comprehension with the difference that they return a
        set instead (so duplicates are filtered out) and are created using two curly braces:
        \mintinline{Python}{s = {item for item in iterable}}

    \subsection{Dictionary Comprehensions}

        Dictionary comprehensions use curly braces, however their first expression is a key-value
        pair:

        \mypy{content/partials/comprehensions/dict_comp.py}
        \mytext{content/partials/comprehensions/dict_comp.out}

    \subsection{Generator Comprehensions}

        Generator comprehensions use parentheses and create and instance of a generator:

        \mypy{content/partials/comprehensions/gen_comp.py}
        \mytext{content/partials/comprehensions/gen_comp.out}

\newpage
\section{Common Built-In Functions} \label{Common Built-In Functions}

    \subsection{Map}

    \subsection{Filter}

    \subsection{Reduce}

\newpage
  % \inputminted{python}{main.py}
\end{document}