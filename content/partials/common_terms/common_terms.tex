\section{Common Terms}
    Before getting into programming and the syntax itself I'd like to explain some common terms
    you'll hear all over the web and maybe work with:

    \begin{indentblock}

        \textbf{CPython} \\
        CPython is a Python interpreter, so it's an implementation of Python.
        It's the most common interpreter available.

        \textbf{PyPy}\footnote{PyPy is a Python interpreter written in Python} \\
        Offering better performance, PyPy is an interpreter as well and often preferred over CPython
        when performance is important.

        \textbf{IPython} \\
        IPython is an interactive Python-Shell.

        \textbf{Jupyter Notebook} \\
        You can think of Jupyter Notebooks as a document storing both text cells and code cells -
        enabling you to describe your code and plot graphs. They're often used for Data Science.
        You can either install Jupyter on your local machine or use cloud-hosted solutions like
        \href{https://colab.research.google.com/}{Google Colab}.

        \textbf{PEP} \\
        The Python Enhancement Proposals - short PEP - are documents with different contents, mainly
        describing language specifications and features Python offers.

        \textbf{PEP8} \\
        One of the most well-known PEPs is PEP8 - the style guide for Python. \\
        Depending on the programming language you're coming from you might have seen different people
        using different code formatting.
        Of course, it's your choice how you format your code - but PEP8 is basically used everywhere and
        I highly encourage you to use it, too.

    \end{indentblock}

    \subsection{Common Modules From The Standard Library}
        The standard library consists of many different modules and the following is only a brief
        overview of the most used ones.\\
        If you're just getting into Python you shouldn't worry about them for you - just remember
        they exist so you can look them up once you need them.

        \begin{indentblock}

            \textbf{os:} interaction with the operating system (e.g. accessing the file system) \\
            \textbf{sys:} information about the file system \\
            \textbf{math:} take a guess
            \textbf{json:} working with JSON
            \textbf{re:} working with regular expressions \\
            \textbf{tkinter:} creation of graphical user interfaces \\
            \textbf{itertools:} provides efficient iterators \\
            \textbf{functools:} provides higher-order functions \\
            \textbf{pickle:} for serialization \\
            \textbf{argparse:} makes it easy to parse and specify command line arguments

        \end{indentblock}

    \subsection{Common Third-Party Modules}
        There are different ways to install third-party Python modules, the most common way is by
        using pip, Python's package manager which is part of your Python distribution
        (if you've downloaded it from \href{https://python.org}{python.org})\\
        The following are some of the most widely used third-party modules and I recommend you to
        look into them:

        \begin{indentblock}

            \textbf{SciPy:} used for scientific and technical computing; contains different modules,
            one of the most popular ones is NumPy.

            \textbf{NumPy:} provides fast (multi-dimensional) arrays with many handy
            features and functions. One NumPy array has exactly one data type.

            \textbf{pandas:} mainly used because of their dataframes; you think of dataframes
            as (e.g. Excel) spreadsheets, so two-dimensional arrays where every column can have a
            different data type.

            \textbf{matplotlib:} used for plotting graphs (e.g. line plots, scatter plots, histograms,
            heat maps)

            \textbf{Flask:} lightweight web framework

            \textbf{Django:} full-stack web framework with many features out of the box

        \end{indentblock}
