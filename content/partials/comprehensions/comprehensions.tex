\section{Comprehensions}

    Often you want to create basic collections following a simple pattern, let's say a list
    containing all the squares of the numbers 1 to 10. You had to create an empty list and then
    fill it within a for loop. \\
    To do this with less code, Python provides different kinds of comprehensions to create
    collections in one line. \\
    The list comprehension will probably be your most used one:

    \subsection{List Comprehensions}

        Let's take the example I just talked about and code it with the knowledge we
        acquired so far:

        \mypy{content/partials/comprehensions/normal_list.py}
        \mytext{content/partials/comprehensions/normal_list.out}

        To achieve the same with a list comprehension:

        \mypy{content/partials/comprehensions/list_comp.py}
        \mytext{content/partials/comprehensions/list_comp.out}

        The \textbf{first expression} - in this case \mintinline{Python}{x**2} -
        is \textbf{added to the list} and this is repeated for every \mintinline{Python}{x} in
        \mintinline{Python}{range(1,11)}, so the numbers 1-10 (as the end of the range -
        the second parameter - is exclusive). \\
        You can also use nested for loops:
        \mintinline{Python}{l = [x for x in item for item in list]}

        \subsubsection{Filtering Elements}

            If you only want to add an element to the list if a certain criteria is met, you can add
            an \mintinline{Python}{if} statement \textbf{at the end of the comprehension}:

            \mypy{content/partials/comprehensions/list_comp_if.py}
            \mytext{content/partials/comprehensions/list_comp_if.out}

        \subsubsection{Ternary Operator}

            You can use the ternary operator in the first expression of the comprehension to decide
            which value to add to the list based on some condition.

            Note that the ternary operator - the \textbf{if at the start of the comprehension} -
            decides on which value to add to the list and the if at the end of the comprehension
            decides on whether the element should even be added to the list or not.

            \mypy{content/partials/comprehensions/list_comp_tern.py}
            \mytext{content/partials/comprehensions/list_comp_tern.out}

            As you can see, the comprehension got quite long and therefore I decided to put
            a line break in.

            \textbf{If your comprehension gets complex you should definitely consider using a normal
            loop instead}, as it's probably easier to read.

    \subsection{Set Comprehensions}

        Set comprehension work just like list comprehension with the difference that they return a
        set instead (so duplicates are filtered out) and are created using two curly braces:
        \mintinline{Python}{s = {item for item in iterable}}

    \subsection{Dictionary Comprehensions}

        Dictionary comprehensions use curly braces, however their first expression is a key-value
        pair:

        \mypy{content/partials/comprehensions/dict_comp.py}
        \mytext{content/partials/comprehensions/dict_comp.out}

    \subsection{Generator Comprehensions}

        Generator comprehensions use parentheses and create and instance of a generator:

        \mypy{content/partials/comprehensions/gen_comp.py}
        \mytext{content/partials/comprehensions/gen_comp.out}
