\subsection{Dictionary}
    A dictionary \textit{(map, associative array)} is a mapping of key-value pairs.

    \begin{indentblock}
        \textbf{Creation:} \mintinline{Python}{d = {}} \sep{,}
        \mintinline{Python}{d = {key: val}} \\
        \mintinline{Python}{d = dict()} \sep{,} \mintinline{Python}{d = dict(iterable)} \\
        You can create a dictionary from an iterable containing indexable items (e.g. lists)
        with a length of 2:

        \mypy{content/partials/collections/dictionary_example.py}
        \mytext{content/partials/collections/dictionary_example.out}

        \textbf{Add Key-Value Pair:} \mintinline{Python}{d[key] = val}

        \textbf{Update (Extend) Set:} \mintinline{python}{d.update(dictionary)} \sep{,}
        \mintinline{python}{d.update(iterable)} \\
        Adds all key-value pairs from the given\textit{dictionary} to the dictionary
        \textit{d}.\\
        You can also extend the dictionary with an iterable containing indexable items
        (e.g. lists) like when creating a new dictionary (see Dictionary Creation)

        \textbf{Access Value:}
        \begin{itemize}
            \bitem{d[key]:} \mintinline{Python}{val = d[key]} \\
            Raises a KeyError if the dictionary does not contain the key.
            \bitem{Get Method:} \mintinline{Python}{val = d.get(key)} \sep{,}
            \mintinline{Python}{val = d.get(key, default=<default value>)} \\
            Returns the default value if the dictionary does not contain the key.
            The default value's default value is None.
        \end{itemize}

        \textbf{Remove Item:}
        \begin{itemize}
            \bitem{Pop:} \mintinline{Python}{d.pop(key)}
            Removes the key-value pair at the given key. Returns the removed value.
            \bitem{Delete Statement:} \mintinline{Python}{del d[index]} \\
            When using the \mintinline{Python}{del} statement the removed value will not be
            returned.
        \end{itemize}

        \textbf{Dictionary Keys:} \mintinline{Python}{d.keys()} \\
        The \mintinline{Python}{d.keys()} methods returns all the keys of a dictionary as a
        dict\_keys
        object. If you would like a normal list instead, just call the list constructor and pass
        the dict\_keys object along as you would do with any iterable:
        \mintinline{Python}{list(d.keys())}

        \textbf{Dictionary Values:} \mintinline{Python}{d.values()} \\
        Returns a dict\_values object. Use \mintinline{Python}{list(d.values())} to get a list
        instead.

        \textbf{Get Key-Value Pairs As List Of Tuples:} \mintinline{Python}{d.entries()} \\
        Returns an iterable with tuples containing the key and value.
        Is especially useful for iterating dictionaries:

        \mypy{content/partials/collections/iterating_dictionaries.py}
        \mytext{content/partials/collections/iterating_dictionaries.out}

        \textbf{Membership Testing:}
        \begin{itemize}
            \bitem{Check For Key:} \mintinline{Python}{key in d}
            \bitem{Check For Value:} \mintinline{Python}{key in d.values()}
        \end{itemize}

        \textbf{Length, Clear:} see \fullref{List}

    \end{indentblock}
