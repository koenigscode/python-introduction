\section{Collections}

    \subsection{List} \label{List}
        A list is an ordered collection, allows duplicate members and is changeable.

        \begin{indentblock}

            \textbf{Creation:} \mintinline{python}{l = []} \sep{,} \mintinline{python}{l = [1,2,3]}
            \sep{,} \mintinline{python}{l = list()} \sep{,} \mintinline{python}{l = list(iterable)}

            You can create a new list using either two square brackets or the list constructor.
            If you want to create a list from an iterable, you need to use the
            \mintinline{python}{list()} constructor:

            \mypy{content/partials/collections/list_example.py}
            \mytext{content/partials/collections/list_example.out}

            \textbf{Append Item:} \mintinline{python}{l.append(item)}

            \textbf{Set Item At Index:} \mintinline{Python}{l[index] = item}

            \textbf{Insert Item At Index:} \mintinline{python}{l.insert(index, item)}

            \textbf{Extend List:} \mintinline{python}{l.extend(iterable)} \\
            \mintinline{python}{l.append([1,2,3])} would add a list containing three items to our list.
            So when we want the items of an iterable (e.g. list) to be added item for item to our list,
            we can extend our list with that iterable.

            \mypy{content/partials/collections/extend_example.py}
            \mytext{content/partials/collections/extend_example.out}

            \textbf{Access Item:} \mintinline{python}{item = l[index]} \sep{,}
            see \fullref{List Slicing} for advanced syntax

            \textbf{List Length:} \mintinline{Python}{len(l)}

            \textbf{Occurences Of Item:} \mintinline{Python}{l.count(item)}

            \textbf{Index Of Item:} \mintinline{Python}{l.index(item)}

            \textbf{Change Item Value:} \mintinline{python}{l[3] = "new value"}

            \textbf{Remove Item By Value:} \mintinline{python}{l.remove(item)} \\
            Removes the item if it is present, otherwise raises KeyError.

            \textbf{Remove Item By Index:}
            \begin{itemize}
                \bitem{Pop:} \mintinline{Python}{l.pop()} \sep{,}
                \mintinline{Python}{l.pop(index)} \\
                Removes the item at the specified position; if no index is given, it removes the last
                item in the list. Returns the removed value.
                \bitem{Delete Statement:} \mintinline{Python}{del l[index]} \\
                When using the \mintinline{Python}{del} statement the removed value will not be
                returned.
            \end{itemize}


            \textbf{Membership Testing:} \mintinline{python}{item in l} \sep{,}
            \mintinline{python}{item not in l}

            \textbf{Reverse List:} \mintinline{Python}{l.reverse()}

            \textbf{Sort List (In Place):} \mintinline{Python}{l.sort()} \sep{,}
            \mintinline{Python}{l.sort(key=<comparison function>, reverse=<boolean>)} \\
            Sorts the list in place (i.e. changes the original list instead of returning a new one)\\
            \textit{key} is an optional function used to compare the elements to each other.

            \textbf{Sort List (As Copy):} \mintinline{Python}{l = sorted(iterable)} \sep{,} \\
            \mintinline{Python}{sorted(iterable, key=<comparison function>, reverse=<boolean>)} \\
            Returns a new, sorted list. Keeps the original iterable unchanged.

            \textbf{Clear List:} \mintinline{python}{l.clear()}
            \\ \\
            Using the \mintinline{Python}{l.append()} and \mintinline{Python}{l.pop()} methods, lists
            can be used as stacks

        \end{indentblock}

        \subsubsection{List Slicing} \label{List Slicing}

    \subsection{Set}
        A set is an unordered (therefore unindexed) and  changeable collection without duplicate
        items.

        \begin{indentblock}

            \textbf{Creation:} \mintinline{python}{s = set()} \sep{,}
            \mintinline{python}{s = set(iterable)} \\
            You \textbf{cannot} create a set using two empty curly braces \textit{\{ \}} as this would
            create an empty dictionary.

            \textbf{Add Item:} \mintinline{python}{s.add(item)} \\
            If the element is already in the list, it won't be added again.

            \textbf{Update (Extend) Set:} \mintinline{python}{s.update(iterable)} \\
            Adds all elements from \textit{iterable} to the set.

            \textbf{Remove Item:} \mintinline{python}{s.remove(item)} \\
            Removes the item if it is present, otherwise raises KeyError.

            \textbf{Discard Item:} \mintinline{python}{s.discard(item)} \\
            Removes the element if it is present, otherwise does nothing.

            \textbf{Membership Testing, Length, Clear:} see \fullref{List}

         \end{indentblock}

    \subsection{Dictionary}
        A dictionary \textit{(map, associative array)} is a mapping of key-value pairs.

        \begin{indentblock}
            \textbf{Creation:} \mintinline{Python}{d = {}} \sep{,}
            \mintinline{Python}{d = {key: val}} \\
            \mintinline{Python}{d = dict()} \sep{,} \mintinline{Python}{d = dict(iterable)} \\
            You can create a dictionary from an iterable containing indexable items (e.g. lists)
            with a length of 2:

            \mypy{content/partials/collections/dictionary_example.py}
            \mytext{content/partials/collections/dictionary_example.out}

            \textbf{Add Key-Value Pair:} \mintinline{Python}{d[key] = val}

            \textbf{Update (Extend) Set:} \mintinline{python}{d.update(dictionary)} \sep{,}
            \mintinline{python}{d.update(iterable)} \\
            Adds all key-value pairs from the given\textit{dictionary} to the dictionary
            \textit{d}.\\
            You can also extend the dictionary with an iterable containing indexable items
            (e.g. lists) like when creating a new dictionary (see Dictionary Creation)

            \textbf{Access Value:}
            \begin{itemize}
                \bitem{d[key]:} \mintinline{Python}{val = d[key]} \\
                Raises a KeyError if the dictionary does not contain the key.
                \bitem{Get Method:} \mintinline{Python}{val = d.get(key)} \sep{,}
                \mintinline{Python}{val = d.get(key, default=<default value>)} \\
                Returns the default value if the dictionary does not contain the key.
                The default value's default value is None.
            \end{itemize}

            \textbf{Remove Item:}
            \begin{itemize}
                \bitem{Pop:} \mintinline{Python}{d.pop(key)}
                Removes the key-value pair at the given key. Returns the removed value.
                \bitem{Delete Statement:} \mintinline{Python}{del d[index]} \\
                When using the \mintinline{Python}{del} statement the removed value will not be
                returned.
            \end{itemize}

            \textbf{Dictionary Keys:} \mintinline{Python}{d.keys()} \\
            The \mintinline{Python}{d.keys()} methods returns all the keys of a dictionary as a
            dict\_keys
            object. If you would like a normal list instead, just call the list constructor and pass
            the dict\_keys object along as you would do with any iterable:
            \mintinline{Python}{list(d.keys())}

            \textbf{Dictionary Values:} \mintinline{Python}{d.values()} \\
            Returns a dict\_values object. Use \mintinline{Python}{list(d.values())} to get a list
            instead.

            \textbf{Get Key-Value Pairs As List Of Tuples:} \mintinline{Python}{d.entries()} \\
            Returns an iterable with tuples containing the key and value.
            Is especially useful for iterating dictionaries:

            \mypy{content/partials/collections/iterating_dictionaries.py}
            \mytext{content/partials/collections/iterating_dictionaries.out}

            \textbf{Membership Testing:}
            \begin{itemize}
                \bitem{Check For Key:} \mintinline{Python}{key in d}
                \bitem{Check For Value:} \mintinline{Python}{key in d.values()}
            \end{itemize}

            \textbf{Length, Clear:} see \fullref{List}

        \end{indentblock}


    \subsection{Tuple}

        Tuples are collections that can, once created, not be changed, meaning you
        cannot add or remove items.

        \begin{indentblock}

            \textbf{Creation:} \mintinline{Python}{t = ()} \sep{,} \mintinline{Python}{tuple()} \\
            \mintinline{python}{t = item1, item2} \sep{,}
            \mintinline{Python}{t = (item1, item2)} \sep{,}
            \mintinline{Python}{t = tuple(iterable)} \\
            You can create tuples containing any amount of items; as you cannot add elements to an
            existing tuple, you will not need to create an empty one very often.

            \textbf{Access Item, Index Of Item, Occurences Of Item, Membership Testing, Length:}
            see \fullref{List}

            Items cannot be added or removed; if an item within the tuple is mutable, however,
            the item itself may be changed.

        \end{indentblock}

        As you can see, a tuple's functionality is quite limited compared to lists. \\
        So why would you use them? Well, for once they are faster than lists and
        \textbf{they enable to you to "return multiple values from a function" thanks to unpacking}:

        \subsubsection{Tuple Unpacking}

            If you got a tuple consiting of \textit{n} elements, you can \textbf{unpack} it and save
            its individual items in \textit{n} (\textit{\textbf{exactly} n}) variables.

            \mypy{content/partials/collections/tuple_unpacking.py}
            \mytext{content/partials/collections/tuple_unpacking.out}

            If you try to unpack a tuple with \textit{n} to less or more than \textit{n} variables,
            an error will be raised. Items not needed are often saved to a dummy variable \texttt{\_}.

            \mypy{content/partials/collections/tuple_error.py}
            \mytext{content/partials/collections/tuple_error.out}

    \subsection{Common Functions For Iterating Collections}
