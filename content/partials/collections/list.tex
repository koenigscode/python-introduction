\subsection{List} \label{List}
    A list is an ordered collection, allows duplicate members and is changeable.

    \begin{indentblock}

        \textbf{Creation:} \mintinline{python}{l = []} \sep{,} \mintinline{python}{l = [1,2,3]}
        \sep{,} \mintinline{python}{l = list()} \sep{,} \mintinline{python}{l = list(iterable)}

        You can create a new list using either two square brackets or the list constructor.
        If you want to create a list from an iterable, you need to use the
        \mintinline{python}{list()} constructor:

        \mypy{content/partials/collections/list_example.py}
        \mytext{content/partials/collections/list_example.out}

        \textbf{Append Item:} \mintinline{python}{l.append(item)}

        \textbf{Set Item At Index:} \mintinline{Python}{l[index] = item}

        \textbf{Insert Item At Index:} \mintinline{python}{l.insert(index, item)}

        \textbf{Extend List:} \mintinline{python}{l.extend(iterable)} \\
        \mintinline{python}{l.append([1,2,3])} would add a list containing three items to our list.
        So when we want the items of an iterable (e.g. list) to be added item for item to our list,
        we can extend our list with that iterable.

        \mypy{content/partials/collections/extend_example.py}
        \mytext{content/partials/collections/extend_example.out}

        \textbf{Access Item:} \mintinline{python}{item = l[index]} \sep{,}
        see \fullref{List Slicing} for advanced syntax

        \textbf{List Length:} \mintinline{Python}{len(l)}

        \textbf{Occurrences Of Item:} \mintinline{Python}{l.count(item)}

        \textbf{Index Of Item:} \mintinline{Python}{l.index(item)}

        \textbf{Change Item Value:} \mintinline{python}{l[3] = "new value"}

        \textbf{Remove Item By Value:} \mintinline{python}{l.remove(item)} \\
        Removes the item if it is present, otherwise raises KeyError.

        \textbf{Remove Item By Index:}
        \begin{itemize}
            \bitem{Pop:} \mintinline{Python}{l.pop()} \sep{,}
            \mintinline{Python}{l.pop(index)} \\
            Removes the item at the specified position; if no index is given, it removes the last
            item in the list. Returns the removed value.
            \bitem{Delete Statement:} \mintinline{Python}{del l[index]} \\
            When using the \mintinline{Python}{del} statement the removed value will not be
            returned.
        \end{itemize}


        \textbf{Membership Testing:} \mintinline{python}{item in l} \sep{,}
        \mintinline{python}{item not in l}

        \textbf{Reverse List:} \mintinline{Python}{l.reverse()}

        \textbf{Sort List (In Place):} \mintinline{Python}{l.sort()} \sep{,}
        \mintinline{Python}{l.sort(key=<comparison function>, reverse=<boolean>)} \\
        Sorts the list in place (i.e. changes the original list instead of returning a new one)\\
        \textit{key} is an optional function used to compare the elements to each other.

        \textbf{Sort List (As Copy):} \mintinline{Python}{l = sorted(iterable)} \sep{,} \\
        \mintinline{Python}{sorted(iterable, key=<comparison function>, reverse=<boolean>)} \\
        Returns a new, sorted list. Keeps the original iterable unchanged.

        \textbf{Clear List:} \mintinline{python}{l.clear()}
        \\ \\
        Using the \mintinline{Python}{l.append()} and \mintinline{Python}{l.pop()} methods, lists
        can be used as stacks

    \end{indentblock}

    \subsubsection{List Slicing} \label{List Slicing}
