\section{Functions}

  I assume you already know about functions; this is what their basic syntax looks like in Python:

  \mypy{content/partials/functions/basic_syntax.py}
  \mytext{content/partials/functions/basic_syntax.out}

  You can also set default parameters by saying
  \mintinline{python}{def my_function(x=1, y=1):}
  so you can call the function with zero, one or two arguments.
  Parameters with a default value are \textbf{optional arguments}, those without are called
  \textbf{positional arguments}.

  Note that \textbf{optional parameters} must be the last arguments in the function definition.
  So \mintinline{python}{def my_function(x=1, y):} is not a valid definition.

  Furthermore, you can \textbf{specify the desired data types} of the function arguments with
  \mintinline{python}{def my_function(x: int, y: int) -> int:}.
  However, everyone's still able to pass variables of any kind to that function,
  meaning \textbf{the data type is not enforced} by any means but just
  for documentation and readability!

  \mypy{content/partials/functions/parameters.py}
  \mytext{content/partials/functions/parameters.out}

  If you wanted to return two items you'd usually return a tuple instead of a list like I did,
  so we'll revisit this function later and improve it when we talk about tuple unpacking.

  As you can see on line 9 you can \textbf{set a parameter by its variable name}
  instead of position.
  This is especially useful when working with functions which can have a huge number of optional
  arguments and you need to only set some specific ones.
  
  \subsection{Arbitrary Argument Lists}

    Sometimes you want to be able to call a function with any number of variables;
    in Python, this is done with *args and you might know this concept from other programming
    languages as 'varargs' (variables arguments).

    Let's take a function \mintinline{python}{my_sum()} as an example:

    \mypy{content/partials/functions/my_sum.py}
    \mytext{content/partials/functions/my_sum.out}

    The function \mintinline{python}{my_sum()} stores all the values it gets in the *args list.
    The * denotes that the variables called 'args' takes all the spare values; The * variables must
    be written after all the positional and optional arguments.
    it is later accessed by just using \mintinline{python}{args}, not \mintinline{python}{*args}!
  
  \subsection{Keyword Arguments}
    Arbitrary argument lists contain a list of spare values, keyword arguments, on the other hand,
    are dictionaries containing spare values set in the function call with the
    \mintinline{python}{key=value} syntax.

    \mypy{content/partials/functions/kwargs.py}
    \mytext{content/partials/functions/kwargs.out}

    Note that \textit{item} is not a keyword argument, as \textit{item} is a normal variable
    from the function definition, however, its value can still be set using the
    \mintinline{python}{key=value} syntax in the1 function call.
    

  
  \subsection{Lambdas}
    Lambdas are a shorthand for one-line functions and created using the
    \mintinline{python}{lambda} keyword. They receive variables and their function body consists
    of only one term, which value also gets returned, therefore an explicit
    \mintinline{python}{return} is not needed.

    \mypy{content/partials/functions/lambda.py}
    \mytext{content/partials/functions/lambda.out}

  \subsection{Common Functions For Iterating Collections}