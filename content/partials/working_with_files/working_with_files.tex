\section{Working With Files}

    \subsection{Opening \& Closing Files}
        The following code is probably pretty self-explanatory:
        using the \mintinline{Python}{read()} function you get a file object
        \footnote{actually TextIOWrapper} and can then read the file's content
        - we will get more into that later.

        Also, make sure to close the file after it is no longer needed.

        \mypy{content/partials/working_with_files/basic_file_example.py}
        \mytext{content/partials/working_with_files/basic_file_example.out}

        As you can see, opening and closing files is quite simple and using the
        \mintinline{Python}{with} \textbf{context manager} makes it even easier as it automatically
        closes the file at the end of it:

        \mypy{content/partials/working_with_files/with_context_manager.py}
        \mytext{content/partials/working_with_files/with_context_manager.out}

    \subsection{Reading Files}

        All of the following methods \textbf{change the SPI}, the stream position indicator, so the
        'current position of the cursor within the file'.

        If you read parts of the file, the SPI moves. If you then use another read method, it will
        start where to old one left off.

        \textbf{Get the current SPI position} \\
        \mintinline{Python}{f.tell()}

        \textbf{Set the SPI position} \\
        \mintinline{Python}{f.seek(idx)} \sep{,}
        idx=0 sets the SPI to the beginning of the file

        \textbf{Read whole file as string} \\
        \mintinline{Python}{f.read() -> str}

        \textbf{Read one line (till next linebreak)} \\
        \mintinline{Python}{f.readline() -> str}

        \textbf{Read whole file as list of lines} \\
        \mintinline{Python}{f.readlines() -> []} \\
        Useful to iterate over a file's lines: \mintinline{Python}{for line in f.readlines(): # ...}

        Example:
        \mypy{content/partials/working_with_files/read_file.py}
        \mytext{content/partials/working_with_files/read_file.out}

        The \mintinline{Python}{end=""} parameter of the \mintinline{Python}{print()} function
        removes the \textit{\textbackslash n} that would usually be printed at the end of the line. \\
        As the read line already contains a line break, I didn't print the extra one from the
        \mintinline{Python}{print()} function.
