\section{Generators}
    Your memory probably can't handle a list of hundreds of thousands of elements.
    Therefore we use generators for tasks like this instead.
    Generators calculate and deliver a value when it's needed.
    
    Consider the following example:

    \mypy{content/partials/generators/gen.py}
    \mytext{content/partials/generators/gen.out}
    
    At first, we define a new generator.
    Functions only return one value, generators, on the other hand, can
    \textbf{yield} multiple values.
    
    Upon calling \mintinline{python}{gen()} on line 14 we get a generator object, so an instance
    of the generator function
    
    When we call \mintinline{python}{next(g)} (with g being the generator object) the generator
    function executes till it reaches the next \mintinline{python}{yield} statement.\\
    Once a value is being yielded, the generator function stops and returns
    the yielded value.\\
    At the next \mintinline{python}{next(g)} the generator function resumes till it reaches
    its next \mintinline{python}{yield}.
    
    
    