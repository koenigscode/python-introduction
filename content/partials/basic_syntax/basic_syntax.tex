\section{Basic Syntax}

    \subsection{If/Else \& For In}
        \mypy{content/partials/basic_syntax/if_else_for_in.py}
        \mytext{content/partials/basic_syntax/if_else_for_in.out}

    \subsection{Ternary Operator (Inline If/Else)}
        The ternary operators works like in other languages, however, you're not using special
        characters like \textit{?} or \textit{:} but the keywords \textit{if} and \textit{else}

        \mypy{content/partials/basic_syntax/inline_if.py}
        \mytext{content/partials/basic_syntax/inline_if.out}
        Note that, however, \mintinline{python}{x = 1 if some_condition} is not valid; you need an
        \mintinline{python}{else} statement.

    \subsection{Chaining Conditions}
        Python enables you to chain conditions:

        \mypy{content/partials/basic_syntax/chaining_conditions.py}
        \mytext{content/partials/basic_syntax/chaining_conditions.out}

    \subsection{Strings And Comments}
        Single line comments start with a \#, docstrings ('multiline comments') comments start and
        end with """. \\
        However, you should use \# for 'normal' comments and """ for docstrings (e.g. explaining the
        purpose of a class or function).

        You can also use """ to create multiline strings:

        \mypy{content/partials/basic_syntax/comments_multiline_strings.py}
        \mytext{content/partials/basic_syntax/comments_multiline_strings.out}

        \subsubsection{String Formatting}
        Concatenating strings with numbers (e.g. \mintinline{python}{"Python" + 101}) results in an
        error. \\
        Gladly, there are some workarounds on how to format your strings:

        \begin{indentblock}

            \textbf{Making everything a string} \\
            \mintinline{python}{"Python" + str(101)}

            \textbf{Modulo Operator} \\
            \mintinline{python}{"Python %s" % (101)} \\
            Calculating \texttt{string \% tuple} works because Python enables you to overwrite
            operators. This has nothing to do with an actual modulo operation, it's just used for
            simplicity. To learn more about operator overloading, take a look at
            \fullref{Object Oriented Programming}

            \textbf{Format Method} \\
            \mintinline{python}{"Python {}".format(101)}

            \textbf{F-Strings} \\
            \mintinline{python}{f"Python {101}"}

        \end{indentblock}


        Here are some examples:

        \mypy{content/partials/basic_syntax/string_formatting_examples.py}
        \mytext{content/partials/basic_syntax/string_formatting_examples.out}

    \subsection{'Main' In Python}
        You probably know the \textit{main()} method from other languages such as Java or C. \\
        Python, on the other hand, is primarily used for scripting, meaning there's a specific file
        to be run rather than a single executable program. \\
        By default, a Python file is executed from top to bottom, no matter whether this is the
        'main' file (the file started e.g. from the command line by using
        \mintinline{bash}{python main.py}) or just a module imported by \textit{main.py}.

        If you want some part of a program only executed if it's the 'main' file, so the
        'entry point' to your program, you can check whether a certain variable called
        \mintinline{Python}{__name__} is equal to \mintinline{Python}{"__main__"}:

        \mypy{content/partials/basic_syntax/main_example.py}
        \mytext{content/partials/basic_syntax/main_example.out}

        If you import this file from another file and execute the latter one, the code within the
        main of the imported file will not be executed.


    \subsection{Exception Handling}
        You probably already know Error or Exception Handling from other object-oriented
        programming languages.

        Generally, we differentiate between two types of errors: syntax errors and exceptions \\
        Syntax errors can - and must be - avoided, exception on the other hand can have different
        origins. \\
        For example you might ask the user for a number - you get the user's input in form
        of a string. You then try to parse the string - however, before trying to parse a number
        from it you don't know whether the user entered a valid number or not.

        There's also a finally statement in Python which is used for clean-up actions like
        closing files. It's always executed, even when an excpetion occured or you used a break,
        continue or return statement.

        \mypy{content/partials/basic_syntax/exception_handling.py}
        \mytext{content/partials/basic_syntax/exception_handling.out}

        If you want to raise (in other languages 'throw') an error, use the raise keyword:
        \mypy{content/partials/basic_syntax/raise_exception.py}
        \mytext{content/partials/basic_syntax/raise_exception.out}
