\section{Modules}

    \subsection{Imports}

    Once you've installed a module (or it's part of the standard library) you can
    easily import it using the import statement.

    Let's consider numpy as an example:\\
    \mintinline{python}{import numpy} enables you to access numpy's objects and modules,
    for instance, the \textit{ones} method (which gives you an array filled with zeros)
    by calling \mintinline{python}{numpy.ones((2,2))}.

        \subsubsection{Alias}

            Most of the time, however, you'll see people use \textit{np} as an alias for numpy.
            Just type \mintinline{python}{import numpy as np} and you're able to call the
            \textit{ones} method with \mintinline{python}{np.ones((2,2))}

            If you just need some specific objects or methods, like the \textit{zeros} or \textit{ones}
            method,
            you can import them by using \mintinline{python}{from numpy import zeros, ones}.
            You can then access the methods by simply saying
            \mintinline{python}{ones((2,2))} and there's no need for the 'numpy' at front.

        \subsubsection{Import \textasteriskcentered}

            Newcomers often tend to use \mintinline{python}{from numpy import *} which
            enables you to use any method or object from the module without prefixing
            it with the module name at the front or importing it explicitly by using its name.\\
            However, you should \textbf{never use a \mintinline{python}{from x import *}}
            as it pollutes your namespace, meaning you've got a problem if
            two or more modules have members with the same name.

            Consider the following example:
            \mypy{content/partials/modules/import_star.py}
            \mytext{content/partials/modules/import_star.out}

            In both modules, a variable \textit{some\_var} was declared and set to different values.
            As you can see in the output, some\_var from b overwrote some\_var from a.
            To avoid these kind of issues, just import modules under a short alias, like
            np for numpy, pd for pandas or tk for tkinter.

            All the methods introduced above also work for submodules, so to use the \textit{pyplot}
            submodule from \textit{matplotlib} as \textit{plt} you can import it using
            \mintinline{python}{import matplotlib.pylot as plt}.

    \subsection{PIP}

        pip is one of the most used package managers for Python and if you've got Python 3.4 or later
        it's already installed on your system.

        Installing a new package/module is as simple as running
        \mintinline{bash}{pip install <package_name>}

        You should then be able to import the newly installed package in Python using the
        \mintinline{Python}{import} statement.

        As an example, let's install \href{https://scikit-learn.org/}{scikit-learn}, a library
        for machine learning, and import it:
        \mybash{content/partials/modules/install_sklearn.sh}
        \mypy{content/partials/modules/install_sklearn.py}
        \mytext{content/partials/modules/install_sklearn.out}
